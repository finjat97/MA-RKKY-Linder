\section{Introduction}
\begin{itemize}
    \item relevancy of superconductors for spintronics
    \item latest advances in experimental set-ups
    \item unconventional superconductors (e.g. high $T_c$ SC \cite{balatsky2006impurity}, he also has an explanation of it incl. sources on page 4)
    \item state of the art regarding RKKY in superconductors
    \item state of the art regarding non-centrosymmetric superconductors
    \item Atousa's work (thesis)
\end{itemize}

The interest in unconventional superconductors is rising because of the recent experimental advances in ???.
They allow to characterize ??? in more detail.
This deepens the knowledge about the spin-state of the cooper pairs, especially of the spin-triplet state.
%% from book: Superconductivity by Asle
Spin-triplet superconductors the spin as well as the charge degree of freedom exhibit superfluid behavior leading to a variety of new phenomena in comparison to spin-singlet superconductors.
Among them are several phases of superconductivity, which is already observed \cite{mackenzie2003superconductivity}, and an expected collective motion of spin or orbital moments with preserved cooper pairs.
Additionally, the state of broken time reversal symmetry (TRS) in spin-triplet superconductors is characterized by the fact that all Cooper pairs within a single superconducting domain have the same sign of their orbital moments.
This allows to describe the TRS state with chirality, which might be used as a quantum bit if it can be externally controlled.

previous work on RKKY in unconventional superconductors \cite{aristov_dn_rkky_nodate}: \newline

\subsection{Structure}
\begin{enumerate}
    \item Introduction
    \item Theory: SWT, BCS, Non-centrosymmetric SC, RKKY, ... (everything that is known entirely)
    \item Results: normal metal, SOC, SC, SOC+SC (state clearly what is known surely and what is new, present own calculations to prove the correctness of the method)
\end{enumerate}