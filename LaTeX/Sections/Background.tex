\section{Theoretical Background}
Hier schon den Hamiltonian einführen und dann später nur noch Lösung und Basistransformation?

\subsection{Normal Metals}
\begin{itemize}
    \item density of states
    \item van-Hoove singularity
    \item spin orbit coupling (different types)
    \item preference of spin orientation due to spin orbit coupling 
    \item Dzyaloshinskii-Moriya interaction 
\end{itemize}

\subsection{Spin Orbit Coupling}


\textcolor{red}{What is this? \newline Which types are there?} \newline

\subsubsection{Rashba type}
The Rashba type of SOC was formulated in 1960 \cite{rashba1960properties} and its general form reads
\begin{equation} \label{eq:general_rashba}
    H_{\gamma} = \sum_{\sigma, \sigma', \langle i,j\rangle} \vb{\gamma}\, \vb{n} \left(\vb{d}_{i,j} \times \Vec{\sigma}_{\sigma, \sigma'} \right) c^{\dag}_{i,\sigma}c_{j,\sigma'}
\end{equation}
with $\sigma, \sigma'$ being spin-indices and $\Vec{\sigma}$ the Pauli-matrix vector. 
There are many different types of SOC and for the moment the type is not specified in any way.
The different forms can be realized by choosing the interaction vector $\vb{d}_{i,j} = \hat{x}(\delta_{i+\hat{x},j}-\delta_{i-\hat{x},j})+\hat{y}(\delta_{i+\hat{y},j}-\delta_{i-\hat{y},j}) +\hat{z}(\delta_{i+\hat{z},j}-\delta_{i-\hat{z},j})$ and the spin orbit field direction vector $\vb{n}$ accordingly.\newline
For an out of plane SOC field $\vb{n} = \hat{z}$ in the framework used here.
Consequently, the Rashba type SOC can be rewritten in k-space as
\begin{align}\label{eq:rashba_SOC}
H_{\gamma} = \sum_{\sigma, \sigma',k} \vb{\gamma}(k_y \hat{x}-k_x\hat{y}) c^{\dag}_{k,\sigma}c_{k,\sigma'}
\end{align}

\subsection{Superconductors}

\begin{itemize}
    \item it is a phase of materials, add link to explanatory video
    \item expelling of magnetic fields (higgs mechanism, mexican hat)
    \item complete loss of electrical resistivity 
    \item type one and type two
    \item conventional vs. unconventional \cite{klam_ludwig_unconventional_2010, balatsky2006impurity}
    \item density of states \cite{sudbo_asle_superconductivity_2004} p.93
    \item LDOS comparison normal metal and superconductor 
\end{itemize}

\subsubsection{BCS Theory}\label{sec:BCS}

\begin{itemize}
    \item cooper pairs
    \item graph with Fermi-surface
    \item different pairing mechanisms
    \item retarding effect
    \item coherence length
    \item proof of lower energy (reference DeltaT just for fun?)
    \item calculation of critical temperature \cite{sigrist_manfred_introduction_nodate} \cite{sudbo_asle_superconductivity_2004}
\end{itemize}
BCS theory is applicable for even frequency superconductors, which are the ones treated here.
\subsection{RKKY interaction} \label{sec:RKKY}
\begin{itemize}
    \item phenomenological explanation
    \item follow paper that discovered it?
    \item does there exist a good book about this?
\end{itemize}
The concept of the Rudermann-Kittel-Kasuya-Yosida (RKKY) interaction was firstly discussed by M. Rudermann and C. Kittel as an explanation for the broadened lines of nuclear spin resonance \cite{rudermann_ma_indirect_1954}. 
It was introduced as the indirect exchange coupling between magnetic moments in a metal via the direct hyperfine interaction with the conduction electrons.
T. Kasuya and K. Yosida expanded this theory to localize inner d-electron interactions \cite{yosida1957magnetic, kasuya1956theory}. \newline
The derivation of different characteristics of the RKKy-interaction takes a non-magnetic metal with two impurity spins, which do not directly interact with each other, as a starting point.
The spin of the conduction electrons of the metal is assumed to interact locally and directly with the impurity spins, which leads to the general expression for the RKKY interaction:
\begin{equation} \label{eq:RKKY_general}
    H_{RKKY} = \sum_{i=1}^2 J \Vec{S}_i \Vec{s}_i
\end{equation}
The itinerant spins can be expressed via fermionic creation and annihilation operators $c^{[\dag]}$ and transformed into k-space.
Based on that expression, the effect of the RKKY-interaction onto the system can be studied via perturbation theory. \newline
To first order in perturbation the correction to the groundstate energy is zero, because the electron system is not spin-polarized.
to second order in perturbation, there is a contribution due to the fact that excited states are taken into account. 
If those excited states correspond to particle-hole excitations, their contribution is not vanishing. 
When using the formalism of spinors and the Pauli matrice, the energy correction to the groundstate energy to second order in perturbation theory reads
\begin{equation}\nonumber
    E_0^{(2)} = \frac{-J^2\hbar^2}{2N^2} \sum_{\substack{k,q \\ i,j}} \Theta_{k,k+q} e^{-iq(r_i-r_j)} \frac{\langle f|\Vec{S}_i \cdot \Vec{S}_j|f\rangle}{\epsilon(k+q)- \epsilon(k}
\end{equation}
where $\Theta$ is the Heaveside-stepfunction, $\epsilon(k)$ is the energy of the unperturbed system, $N$ the total number of particles and $|f\rangle$ is the spin state of the itinerant electrons. \newline
Therefore the coupling constant $J$ is
\begin{equation} \label{eq:J_general}
    J_{ij}^{RKKY} = \frac{J^2\hbar^2}{2N^2} \sum_{k,q,m_s} \sum_{i,j=1}^2 \Theta_{k,k+q} e^{-iq(r_i-r_j)} \frac{\langle f|\Vec{S}_i \cdot \Vec{S}_j|f\rangle}{\epsilon(k+q)- \epsilon(k}
\end{equation}
which shows an oscillatory behavior as a function of the separation $(r_i-r_j)$ of the impurity spins. \newline
The oscillatory nature of this interaction can also be understood based on the electron density between the two impurity spins.
When treating the impurity spins as ferromagnetic layers that enclose a non-magnetic layer representing the normal metal, it becomes clear that the wave function of the electron depends on the free plane-wave and a reflected wave.
The probability of finding an electron at a certain position is therefore $|\Pi(x)|^2 = |exp(ikx) +R exp(-ikx)|^2 = 1 + R^2 + 2Rcos(2kx)$ with the reflection coefficient $R$.
Therefore the spin-density of the electrons odes also vary proportional to $cos(x)$.
Since the electrons are the carries of the spin information that facilitate the indirect interaction between the fixed spins, this interaction is also of oscillatory nature. \newline
\textcolor{red}{Find a graph to show oscillatory behavior. And a graph to schematically show what kind of interaction we are talking about.}

