\subsection{RKKY as Perturbation}
The fermionic, itinerant spins in the non-centrosymmetric superconductor described above are able to interact locally with impurities spins, which leads to the RKKY interaction between the impurity spins.
This interaction is introduced in Sec. \ref{sec:RKKY} and is of second order in perturbation theory.
Therefore the local spin-interaction term
% \begin{equation}\label{eq:RKKYhamiltonian}
%     H_{RKKY} = J \sum_i \vb{S}_i \vb{s}_i
% \end{equation}
% is going to be treated as a perturbation to the system.
% The classical spin $\vb{S}_i$ describes a local spin-1/2-impurity placed at position $\vb{r}_i$, while $\vb{s}_i$ is the quantum-mechanical spin-operator for the mobile spins.
% The summation runs over all impurity spin positions and the interaction strength $J$ is assumed to be constant. \newline
% The spin-operator is represented by
% \begin{align}
%     \vb{s}_i &= \sum_{\sigma, \sigma'} \Vec{\sigma}_{\sigma, \sigma'} c^{\dag}_{i,\sigma}c_{i,\sigma'}
% \end{align}
% and can be Fourier-transformed.
% The perturbation term therefore becomes

\begin{equation} \nonumber
    H_{RKKY} = \sum_{\sigma, \sigma', k, k'} \sum_i \frac{J}{N}e^{i(k-k')r_i}(\vb{S}_i\cdot \Vec{\sigma}_{\sigma, \sigma'})c^{\dag}_{k,\sigma}c_{k',\sigma'}
\end{equation}
is treated as a perturbation to the system and firstly is transformed into helicity basis by means of Eq. \eqref{eq:helicitytrafo}:
\begin{align} \nonumber
    H_{RKKY} &= \sum_{k, k', i, \lambda, \lambda'} T^{ \gamma}_{i, k, k', \lambda, \lambda'} b^{\dag}_{k,\lambda}b_{k',\lambda'}
\end{align}
where $T_{\gamma, k, k', i, \lambda, \lambda'}$ contains all pre-factors. 
The exact expression for $T$ can be found in App. \ref{app:basistrafo}.\newline
After another transformation using Equation \eqref{eq:d_basis_def}, the RKKY-interaction reads in the eigenbasis of the unperturbed Hamiltonian: 
\begin{align}
    &H_{RKKY} = \sum_{i, k, k',\lambda, \lambda'} T^{ \gamma}_{i, k, k', \lambda, \lambda'} \\ \nonumber 
    &\left(
    \eta^{\dag}_{k, \lambda}\eta_{k',\lambda'} d^{\dag}_{k,\lambda}d_{k',\lambda'}
    + \nu_{k,\lambda} \nu^{\dag}_{k',\lambda'} d_{-k,\lambda}d^{\dag}_{-k',\lambda'}
    - \nu_{k,\lambda}\eta_{k',\lambda'} d_{-k,\lambda}d_{k',\lambda'} 
    - \eta^{\dag}_{k,\lambda} \nu^{\dag}_{k',\lambda'} d^{\dag}_{k,\lambda}d^{\dag}_{-k',\lambda'} \right)
\end{align}
The interaction between impurity spins and superconductor aka. itinerant spins is approximated to be into one direction only. 
The changes within the superconductor because of the impurities is neglected and only the influence of the superconducting environment onto the configuration of the impurity spins is investigated.
This simplification can be justified by the low density of impurity spins in the superconducting system.

